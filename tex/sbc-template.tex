\documentclass[12pt]{article}

\usepackage{sbc-template}
\usepackage{graphicx,url}
\usepackage[utf8]{inputenc}
\usepackage[brazil]{babel}

     
\sloppy

\title{Aplicação dos algoritmos de árvore de decisão e rede neural \textit{feed-forward} para a classificação de dados do jogo DotA2}

\author{Eduardo Gil S. Cardoso\inst{1}, Gabriela S. Maximino\inst{1}, Igor Matheus S. Moreira\inst{1}}


\address{Instituto de Ciências Exatas e Naturais -- Faculdade de Computação\\
  Universidade Federal do Pará -- Belém, PA -- Brasil
  \email{\{eduardo.gil.s.cardoso,gabriela.maximino,igor.moreira\}@icen.ufpa.br}
}

\begin{document} 

\maketitle

\begin{abstract}
  This article presents the application of the decision tree and feed-forward artificial neural network algorithms in a database containing information about DotA game matches, aiming at data classification. This report is part of the deliverable associated to the task proposed by professor Reginaldo Cordeiro dos Santos Filho for the Artificial Intelligence course, taught under the Computer Science Bachelor’s degree program at the Federal University of Pará.
\end{abstract}
     
\begin{resumo} 
  Este artigo apresenta a aplicação dos algoritmos de árvore de decisão e rede neural artificial \textit{feed-forward} em uma base de dados contendo informações sobre partidas do jogo DotA, visando a classificação dos dados. Este trabalho é parte do entregável relativo à tarefa proposta pelo Prof. Dr. Reginaldo Cordeiro dos Santos Filho para a disciplina de Inteligência Artificial, ministrada sob o curso de Bacharelado em Ciência da Computação na Universidade Federal do Pará.
\end{resumo}


\section{Introdução}\label{sec:intro}
A descoberta de conhecimento em base de dados (\textit{Knowledge Discovery in Databases} ou KDD) é o processo de descoberta de padrões válidos, potencialmente úteis e entendíveis a partir de dados. De modo geral, o KDD divide-se em 5 etapas: seleção; pré-processamento; transformação; mineração de dados; e interpretação de resultados \cite{fayyad}. Na etapa de mineração de dados, tem-se diversas tarefas; dentre elas, a classificação, responsável por rotular os dados em classes previamente definidas.

Nesse contexto, o terceiro trabalho da disciplina Inteligência Artificial propõe a realização do KDD em uma base de dados qualquer advinda do site \textit{UCI Machine Learning Repository}, considerando a utilização dos algoritmos de árvore de decisão e rede neural artificial para a tarefa de classificação na etapa de mineração de dados. Para tal, foi escolhido um repositório contendo informações sobre partidas do jogo DotA2, cuja classificação tenta predizer qual time ganhou a partida com base nos atributos considerados.

Diante disso, as seções subsquentes estão dividas da seguinte forma: a Seção \ref{sec:descricao} apresenta a descrição da base de dados selecionada; a Seção \ref{sec:metodologia} descreve o processo de realização do trabalho; a Seção \ref{sec:resultados} apresenta os resultados da aplicação dos algoritmos; a Seção \ref{sec:analise} apresenta a comparação entre os dois algoritmos; por fim, a Seção \ref{sec:conclusao} sintetiza o trabalho e apresenta as considerações finais.

\section{Descrição da base de dados}\label{sec:descricao}

\section{Metodologia do trabalho}\label{sec:metodologia}

\section{Resultados}\label{sec:resultados}

\section{Análise comparativa}\label{sec:analise}

\section{Conclusão}\label{sec:conclusao}



\bibliographystyle{sbc}
\bibliography{sbc-template}

\end{document}
